\documentclass{article}
\usepackage[T2A,T1]{fontenc}
\usepackage[utf8]{inputenc}
\usepackage[russian]{babel}
\usepackage[left=1cm,right=1cm,top=2cm,bottom=2cm]{geometry}
\usepackage{amsmath}
\usepackage{graphicx}
\graphicspath{{../data/images/}}
\usepackage{listings}
\usepackage[font=small,labelfont=bf]{caption}
\usepackage{csvsimple}
\usepackage{adjustbox}
\usepackage{verbatim}
\usepackage{hyperref}

\title{Параллельное программирование\\}
\author{ИИКС ИБ\\Б19-505\\Голигузов Алексей}
\date

\begin{document}

\maketitle
\newpage
\tableofcontents

\newpage
\section{Исследование}
Реализация системы шифрования, основанной на идее, предложенной в \href{https://www.fq.math.ca/Papers1/50-2/HamlinWebb.pdf}{статье}.

\newpage
\section{Алгоритм}
Базовая часть алгоритма состоит из следующих частей:
\begin{enumerate}
    \item Генерация последовательности по сигнатуре реккуренты; в качестве базы
    реккуренты берется набор из n единиц(1), где n - длина сигнатуры
    \item Генерация базовых подстрок(блоков) по сигнатуре
    \item Обновление множества базовых подстрок за счет полной параллельной
    проекции их на строку, длиной, равной размеру блока
    \item Генерация карты представления чисел
\end{enumerate}

Шифрование:
\begin{enumerate}
    \item Каждый элемент во входном тексте представляеся в виде суммы элементов(жадным алгоритмом)
    карты
    \item Данное представление упаковывается в конечное представление - строку, длиной, равной размеру блока
\end{enumerate}

Дешифрование:
\begin{enumerate}
    \item Входной текст делится на куски, длиной в размер блока
    \item Каждый кусок скалярно перемножается с посчитанной рекурентой, что дает значение исходного символа
\end{enumerate}

Методика распараллеливания заключается в том, что мы можем обрабатывать каждый символ(при шифровании) 
и каждый блок(при дешифровании) независимо. Также некоторые внутренние операции также могут быть выполнены
параллельно(скалярное перемножение, упаковка и другие).

\newpage
\section{Аппаратные характеристики тестового стенда}
\verbatiminput{system_info.txt}

\newpage
\section{Программная конфигурация}
Использованный язык программирования: C++\\
Компилятор: g++ (10.3.0)\\
Распараллеливание: OpenMP (201511)\\
Собиралось с флагами: -fopenmp -O3 -std=c++20

\newpage
\section{Эксперимент}
Были проведены эксперименты для 5 различных размеров блока(16, 32, 64, 128, 256).
Каждый эксперимент состоял из 10000 тестов, каждый тест для каждого кол-ва потоков(от 1 до 16).
Внутри каждого теста сигнатура и входные данные генерировалиль случайно.
В качестве сигнатуры реккуренты использовалась рандомно сгенерированная валидная сигнатура длины 2,
величины коеффициентов - целые неотрицательные числа, не превышающие 9.
Длина входного текста - фиксированная, 512 символов.

\newpage
\section{Графики}
\begin{minipage}{0.48\linewidth}
    \includegraphics[width=\linewidth]{enc_time.png}
    \captionof{figure}{Время}
\end{minipage}
\hfill
\begin{minipage}{0.49\linewidth}
    \includegraphics[width=\linewidth]{dec_time.png}
    \captionof{figure}{Время}
\end{minipage}

\begin{minipage}{0.48\linewidth}
    \includegraphics[width=\linewidth]{enc_accel.png}
    \captionof{figure}{Ускорение}
\end{minipage}
\hfill
\begin{minipage}{0.48\linewidth}
    \includegraphics[width=\linewidth]{dec_accel.png}
    \captionof{figure}{Ускорение}
\end{minipage}

\begin{minipage}{0.49\linewidth}
    \includegraphics[width=\linewidth]{enc_eff.png}
    \captionof{figure}{Эффективность}
\end{minipage}
\hfill
\begin{minipage}{0.49\linewidth}
    \includegraphics[width=\linewidth]{dec_eff.png}
    \captionof{figure}{Эффективность}
\end{minipage}

\newpage
\section{Заключение}
Получения реализация системы, в основе которой лежит идея из статьи. На основаниях графиков, можно сделать выводы, что
распараллеленная версия дешифрования была ускорена в 2-3 раза по времени.

Для алгоритма шифрования, при кол-ве потоков > 9 имеем заметное ускорение в 2-3 раза, причем,
чем больше размер блока, тем больше ускорение.

\newpage
\section{Приложение 1}
\lstinputlisting[language=C++]{crypto.cpp}

%\newpage
%\section{Приложение 1a}
%\lstinputlisting[language=Python]{graph.py}


\end{document}